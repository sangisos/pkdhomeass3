\documentclass[12pt, a4paper]{article}
\usepackage[utf8]{inputenc}
\usepackage[swedish]{babel}
\usepackage{fullpage}
\usepackage{hyperref}
\usepackage{amsmath}
\usepackage{amssymb}
\usepackage{mathtools}
\usepackage[parfill]{parskip}

\title{Time complexity analasys of card cryptology functions}
\author{Anna Normark, 940319-1308 and Alexander Andersson, 860616-1530}

\begin{document}
\maketitle



\section{preprocess}
Our preprocess is linear as it only calls our auxiliary function preprocess'.

\subsection{preprocess'}
Let n be the length of the list cl. The worst case is if every character in cl is a alpha character, and the length of cl modulus 5 is 1, as we then have to add 4 X:s.

This function spends most of it's time in the fifth case, where it removes the first character c of the char list cl, checks if it is a alpha character (constant time), if it's not, recurse with the char removed from cl, else make it uppercase (constant time), add it to chunk (constant time) and removes 1 from chunkSize (constant time), then recurses on the new data.

For every fifth (actually every chunkSize:th) character that is added to chunk, we enter the forth case in preprocess', where we reset the chunkSize, send a empty list as new chunk, revers the old chunk (linear to chunkSize, in this case five), and adds the reversed chunkSize to the list full (constant time).

When we reach the end of cl, we add X:s, but as it is such a small part of the total it is negligible.

At last the total list of chunks is reversed, linear to the length of cl divided by chunkSize, in this case 5.

All in all, this gives us:

$$C(n) = C(n-1) + f(n)$$
$$f(n) = \Theta(1)$$
$$C(n) = \Theta(n)$$ for case five,


$$C(n) = C(n - chunkSize) + f(n)$$
$$f(n) = \Theta(chunkSize)$$
$$C(n) = \Theta(n)$$ for case four, and

$$C(n) = f(n)$$
$$f(n) = \Theta(n/5)$$
$$C(n) = \Theta(n/5) = \Theta(n)$$ for either of the first two cases.

And thus we get the whole function:
$$C(n) = 3 * \Theta(n) = \Theta(n)$$

\section{encrypt and decrypt}
Both encrypt and decrypt have the same time complexity as enDecrypt since both functions only call enDecrypt with an operation.

\subsection{enDecLetter}
enDeckLetter is a constant function since all operators and auxiliary functions are constant.

\subsection{enDecrypt}
Let n be the length of the total number of characters in l (length l * chunkSize). We assume the deck of cards to be of constant size.

enDecrypt calls keystream with n (linear to n), and calls split (linear to n) on the result. It then calls ListPair.map (linear on chunkSize) with enDecLetter (constant) on each chunk of each list l and the split keystream, which it gets from calling ListPair.map (linear on l) on the list l and the split keystream. This makes the whole function linear in relation to n.

$$C(n) = \Theta(\text{length l}*\text{chunkSize}) + \Theta(n) + \Theta(n) = 3 * \Theta(n) = \Theta(n)$$

\section{keystream}
If we assume that the deck of cards is fixed according to the third point in complexity analysis in assignment 3 clarifications. Let n be the argument to keystream.
keystream is a linear function because it uses keyedDeck, which will be constant according to assumption, and calls the function keystream' with n which is linear to n, and therefor keystream is linear in relation to n.

$$C(n) = \Theta(n) + \Theta(1) = \Theta(n)$$

\subsection{moveJoker}
MoveJoker is a linear function since all the operations are linear and the function is not run recursively.

\subsection{moveJokerAOneCard and moveJokerBTwoCards}
Let m be the number of cards before the searched joker. In worst case the joker is the last card and m is the number of cards-1.
Both moveJokerAOneCard and moveJokerBTwoCards is linear because it only uses constant operators recursively until the last step, where it uses moveJoker, which also is a linear function and therefor moveJokerAOneCard and moveJokerBTwoCards is linear.

$$C(0) = \Theta(m)$$
$$C(m) = C(m-1) + f(m)$$
$$f(m) = \Theta(1)$$
Which through Theorem 1 gives us the closed form
$$C(m) = \Theta(m) + \Theta(m) = \Theta(m)$$

\subsection{tripleCut}
Let m be the length of the deck. In worst case scenario both jokers are at the end of the deck, as we then have to traverse the whole deck.

The function is linear since it will only be run recursively one time each call, according to Theorem 1.

$$C(0) = C(n-1) + f(n)$$
$$C(n) = C(n-1) + f(n)$$
$$f(n) = \Theta(1)$$
Which through Theorem 1 gives us the closed form
$$C(n) = \Theta(n)$$

\subsection{countCut}
CountCut is a linear function since all the operations are linear and it does not run recursively.

\subsection{findOutputLetter}
FindOutputLetter is a linear function since all the operators it uses is constant except for List.nth which is linear, and therefor in worst case scenario the head of the deck is a joker.

\subsection{keystream'}
keystream' is linear in relation to the chars n it will return, and returning each char is linear in relation to the size of the deck m. But according to the third statement in complexity analysis of assignment3 clarifications we can consider the number of cards fixed, and therefor returning each char will be constant. We also make the simplification of not counting the times we hit a joker and have to recalculate the current char (this happens 419 times while getting 10000 characters, less then 5\% of the times).

$$C(0) = \Theta(1)$$
$$C(n) = C(n-1) + f(n)$$
$$f(n) = \Theta(m) = \Theta(1)$$
We can then conclude that keystream' is a linear function.
$$C(n) = \Theta(n)$$
\end{document}
